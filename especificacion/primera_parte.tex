\documentclass[10pt,a4paper]{article}

\input{AEDmacros}
\usepackage{caratula} % Version modificada para usar las macros de algo1 de ~> https://github.com/bcardiff/dc-tex


\titulo{Especificaci\'on y correcci\'on de problemas}
\fecha{\today}

\materia{Algoritmos y Estructuras de Datos}
\grupo{DMSCXIDCOPMHURGQJPDP}

\integrante{Delicia, Felipe Nahuel}{002/01}{delicia4581@gmail.com}
\integrante{Bagini, Jeremias Agustin}{618/20}{jerebagini@gmail.com}
\integrante{Bol\'ivar, Carlos}{990/23}{carlosbolivarp24@gmail.com}
\integrante{De La Pina, Lucas Le\'on}{924/22}{lldelapina@gmail.com}
% Pongan cuantos integrantes quieran

% Declaramos donde van a estar las figuras
% No es obligatorio, pero suele ser comodo
\graphicspath{{../static/}}

\begin{document}

\maketitle

\section{Especificaci\'on}
\subsection{redistribucionDeLosFrutos}

\begin{proc}{redistribucionDeLosFrutos}{\In recursos : \TLista{\float}, \In cooperan : \TLista{\bool}}{\TLista{\float}}

\requiere{\longitud{recursos} = \longitud{cooperan}}
% \requiere{(\longitud{recursos} > 0) \wedge (\longitud{cooperan} > 0)}
% Revisar si conviene la version de Jere

\asegura{recursosPosDistribucion(res, recursos, cooperan) \wedge (\longitud{res} = \longitud{recursos})}
\hfill \break

\pred{recursosPosDistribucion}{\In result : \TLista{\float} ,\In recursos : \TLista{\float}, \In cooperan : \TLista{\bool}}
{\paraTodo[unalinea]{i}{\ent}{(0\leq i < \longitud{result}) \yLuego (cooperan[i] = \True) \implicaLuego  result[i] = \frac{fondoComun(recursos, cooperan)}{\longitud{recursos}} } 
\newline \wedge \newline 
\paraTodo[unalinea]{j}{\ent}{(0\leq j < \longitud{result}) \yLuego (cooperan[j] = \False) \implicaLuego  result[j] = recursos[j] + \frac{fondoComun(recursos, cooperan)}{\longitud{recursos}} } 
} 
\hfill \break
\aux{fondoComun}{\In recursos : \TLista{\float}, \In cooperan : \TLista{\bool}}{\float}\hfill {
$\sum_{k=0}^{\longitud{recursos}-1}$
(\IfThenElse{cooperan[k] = \True}{recursos[k]}{0})
}

\end{proc}

\subsection{Trayectoria de los frutos individuales a largo plazo}

\begin{proc}{trayectoriaDeLosFrutosIndividualesALargoPlazo}{\Inout trayectorias : \TLista{\TLista{\float}}, \In cooperan : \TLista{\bool}, \In apuestas : \TLista{\TLista{\float}}, \In pagos : \TLista{\TLista{\float}}, \In eventos : \TLista{\TLista{\nat}}}{}

\requiere{(\longitud{trayectorias} = \longitud{cooperan}) \wedge (\longitud{apuestas} = \longitud{pagos})}
% Requieres q pense q habria q sumar, debatirlo antes de ponerlos en la version final
\requiere{\longitud{trayectorias} \leq 1}

\requiere{\longitud{trayectorias} = \longitud{cooperan} = \longitud{apuestas} = \longitud{pagos} = \longitud{eventos}}

\requiere{\paraTodo[unalinea]{i}{\ent}{0 \leq i < \longitud{apuestas}}{\implicaLuego \sum\limits_   {j=0}^{\longitud{apuestas[i]}-1}apuestas[i][j] = 1 \land apuestas[i][j] > 0}}

\requiere{\paraTodo[unalinea]{k}{\ent}{0 \leq k < \longitud{trayectorias}}{ \implicaLuego trayectorias[i]} = 1}

\requiere{\paraTodo[unalinea]{l}{\ent}{0 \leq l < \longitud{pagos}}{ \implicaLuego elementosPositivos(l)}}

\requiere{\paraTodo[unalinea]{m}{\ent}{0 \leq m < \longitud{eventos}}{\implicaLuego tieneEventosPosibles(eventos[m],pagos[m]}}

\requiere{\paraTodo[unalinea]{n}{\ent}{0 \leq n < \longitud{apuestas}}{\implicaLuego \longitud{apuestas[n] = pagos[n]}}}

%Falta pedir que la longitud interna de todas las listas de listas sean iguales y escribir los preds y auxiliares de estos requiere

%Hay un desacuerdo con la interpretacion: por un lado, es posible pensar que trayectorias[0] = lista(recursosDeJugadorEnPosicion0). Es posible que sea trayectorias[0] = lista()
% \asegura{\longitud{trayectorias} \geq 
% \longitud{old(trayectorias)}}

\asegura{\paraTodo[unalinea]{k}{\ent}{0 \leq k < \longitud{trayectorias} \implicaLuego \longitud{trayectorias[k] \geq \longitud{old(trayectorias[k])}}}}
\asegura{
\paraTodo[unalinea]{i}{\ent}{
0 \leq i < \longitud{eventos} \implicaLuego 
\paraTodo[]{j}{\ent}{0 \leq j < \longitud{eventos[i]} \implicaLuego trayectorias[i][j+1] = \IfThenElse{(coopera[i] = \False)}{\\ pagoParaJugadoriDelEventoj(trayectoria, apuestas, pagos, eventos, i, j) \\ + 
frutosFondoComun(trayectoria, apuestas, pagos, eventos, cooperan, i, j)\\}{frutosFondoComun(trayectoria, apuestas, pagos, eventos, cooperan, i, j)} }
}
}
\hfill \break
\aux{pagoParaJugadoriDelEventoj}{\In trayectorias : \TLista{\TLista{\float}}, \In apuestas : \TLista{\TLista{\float}}, \In pagos : \TLista{\TLista{\float}}, \In eventos : \TLista{\TLista{\nat}}, \In i : \ent, \In j : \ent}{\float}{
trayectoria[i][j]*apuestas[i][\textbf{eventos[i][j]}]*pagos[i][\textbf{eventos[i][j]}]
}
\hfill \break
\aux{frutosFondoComun}{\In trayectorias : \TLista{\TLista{\float}}, \In apuestas : \TLista{\TLista{\float}}, \In pagos : \TLista{\TLista{\float}}, \In eventos : \TLista{\TLista{\nat}}, \In cooperan : \TLista{\bool}, \In i : \ent, \In j : \ent}{\float}{\frac{fondoComunDelEventoj(trayectoria, apuestas, pagos, eventos ,cooperan, j)}{\longitud{cooperan}}}
\hfill \break
\aux{fondoComunDelEventoj}{\In trayectorias : \TLista{\TLista{\float}}, \In apuestas : \TLista{\TLista{\float}}, \In pagos : \TLista{\TLista{\float}}, \In eventos : \TLista{\TLista{\nat}}, \In cooperan : \TLista{\bool}, \In j : \ent}{\float}{
\\ \sum_{i=0}^{|cooperan|-1} (\IfThenElse{cooperan[i] = \True}{pagoParaJugadoriDelEventoj(trayectoria, apuestas, pagos, eventos, i, j)}{0})
}
\end{proc}

\subsection{Trayectoria Extraña Escalera}

\begin{proc}{trayectoriaExtrañaEscalera}{\In trayectoria : \TLista{\float}}

\requiere{\longitud{trayectoria} > 0}

\asegura{res = \True \iff unicoMaximoLocal(trayectoria)}
\hfill \break
\pred{unicoMaximoLocal}{\In trayectoria : \TLista{\float}}
{(\longitud{trayectoria} = 1 )\newline \lor \newline (\longitud{trayectoria} = 2  \land (trayectoria[0] \neq trayectoria[1])) \newline \lor \newline (\longitud{trayectoria} > 2 \land cantidadMaximosLocales(trayectoria) = 1)}
\hfill \break
\aux{cantidadMaximosLocales}{\In trayectoria : \TLista{\float}}{\float}{(\IfThenElse{trayectoria[0] > trayectoria[1]}{1}{0})\newline+\newline(\IfThenElse{trayectoria[\longitud{trayectoria} - 1] > trayectoria[\longitud{trayectoria} - 2]}{1}{0})\newline+\newline(\IfThenElse{\sum\limits_{i=1}^{\longitud{trayectoria} - 2} trayectoria[i - 1] < trayectoria[i] \land trayectoria[i + 1] < trayectoria[i])}{1}{0}}

\end{proc}

\subsection{Individuo decide si cooperar o no}
\begin{proc}{individuoDecideSiCooperarONo}{\In individuo : \nat, \In recursos : \TLista{\float}, \Inout cooperan : \TLista{\bool}, \In apuestas : \TLista{\TLista{\float}},  \In pagos : \TLista{\TLista{\float}},  \In eventos : \TLista{\TLista{\nat}}}{}

\requiere{ 0 \leq individuo < \longitud{recursos}}
\requiere{\longitud{recursos} = \longitud{cooperan} = \longitud{apuestas} = \longitud{pagos}}

\asegura{cooperan[individuo] = true \Leftrightarrow \\(recursosFinalesSiCoopera(individuo, recursos, cooperan, apuestas, pagos, eventos) \geq \\recursosFinalesSiDeserta(individuo, recursos, apuestas, pagos, eventos))}
\hfill \break
\aux{recursosFinalesSiCoopera}{\In i : \nat, \In recursos : \TLista{\float},  \In apuestas : \TLista{\TLista{float}},  \In pagos : \TLista{\TLista{float}},  \In eventos : \TLista{\TLista{\nat}}, \In cooperan : \TLista{\bool}}{\float}\\
{frutosFondoComunTrasEventoj(recursos,coopera,apuestas,pagos,eventos,\longitud{eventos[individuo]})}

\hfill \break
\aux{recursosFinalesSiDeserta}{\In i : \nat, \In recursos : \TLista{\float},  \In apuestas : \TLista{\TLista{\float}},  \In pagos : \TLista{\TLista{\float}},  \In eventos : \TLista{\TLista{\nat}}, \In cooperan : \TLista{\bool}}{\float}
{\sum_{j=0}^{\longitud{eventos[i]}-1} \textbf{if} j=0\; \textbf{then}\\  recursos[i]*productoriaPagos(apsts,pagos,evtos,i,j)\;\textbf{else} \\productoriaPagos(apsts,pagos,evtos,i,j)*frutosFondoComunTrasEventoj(rcsos,apsts,pagos,evtos,cooperan,j-1) \;\textbf{fi}}

\hfill \break
\aux{productoriaPagos}{\In recursos : \TLista{\float},\In apuestas : \TLista{\TLista{\float}}, \In pagos : \TLista{\TLista{\float}}, \In eventos : \TLista{\TLista{\nat}}, \In i : \ent, \In j : \ent}{\float}{
(\prod_{j'=j}^{\longitud{eventos[i]}-1}{apuestas[i][eventos[i][j']]*pagos[i][eventos[i][j']])}
}
\hfill \break
\aux{frutosFondoComunTrasEventoj}{\In recursos : {\TLista{\float}}, \In apuestas : \TLista{\TLista{\float}}, \In pagos : \TLista{\TLista{\float}}, \In eventos : \TLista{\TLista{\nat}}, \In cooperan : \TLista{\bool}, \In cantEventos : \ent}{\float}
{\prod_{j=0}^{cantEventos-1}\\
{\IfThenElse{j = 0}{\frac{fondoComunj(rcsos,apstas,pagos,evtos,cooperan,j,true)}{\longitud{cooperan}}}{\frac{fondoComunj(rcsos,apstas,pagos,evtos,,cooperan,j,false)}{\longitud{cooperan}}}}}


\hfill \break
\aux{fondoComunj}{\In recursos : {\TLista{\float}}, \In apuestas : \TLista{\TLista{\float}}, \In pagos : \TLista{\TLista{\float}}, \In eventos : \TLista{\TLista{\nat}}, \In cooperan : \TLista{\bool}, \In j : \ent, \In primer-evento : \bool}{\float}{
\\ \sum_{i=0}^{|cooperan|-1} \\(\IfThenElse{cooperan[i] = \True} {recursosParaJugadoriTrasEventoj(rcsos, apsts, pagos, evtos, i, j, primer-evento}{0})
}
\hfill \break 
\aux{recursosParaJugadoriTrasEventoj}{\In recursos : \TLista{\float}, \In apuestas : \TLista{\TLista{\float}}, \In pagos : \TLista{\TLista{\float}}, \In eventos : \TLista{\TLista{\nat}}, \In i : \ent, \In j : \ent, \In primer-evento : \bool }{\float}{ \IfThenElse{primer-evento = true}{recursos[i]*apuestas[i][\textbf{eventos[i][j]}]*pagos[i][\textbf{eventos[i][j]}]}{apuestas[i][\textbf{eventos[i][j]}]*pagos[i][\textbf{eventos[i][j]}]}}
\end{proc}


\subsection{Individuo actualiza apuesta}
\begin{proc}{individuoActualizaApuesta}{\In individuo : \nat, \In recursos : \TLista{\float}, \In cooperan : \TLista{\bool}, \Inout apuestas : \TLista{\TLista{\float}},  \In pagos : \TLista{\TLista{\float}},  \In eventos : \TLista{\TLista{\nat}}}{}
\requiere{}
\asegura{apuestasOptimas(individuo,recursos,cooperan,apuestas,pagos,eventos)}
\hfill \break
\pred{apuestasOptimas}{\In i : \ent, \In recursos : \TLista{\float}, \In cooperan : \TLista{\bool}, \In apuestas : \TLista{\TLista{\float}}, \In pagos : \TLista{\TLista{\float}}, \In eventos : \TLista{\TLista{\nat}}}
{\paraTodo[unalinea]{j}{\ent}{0 \leq j < \longitud{eventos[i]-1}
\implicaLuego apuestas[i][eventos[i][j]] = 1
}
}

\end{proc}




\end{document}